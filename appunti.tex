\section{Appunti}
\subsection{Incontro1}
\textbf{Problema:} esiste una possibilità di censura su determinati contenuti 

\textbf{Tesi:} esiste una differenza tra bias espressi da modelli americani vs modelli cinesi, ovvero il Paese di sviluppo di un modello influisce sulla censura del modello stesso?

Cose da fare:
\begin{itemize}
    \item Dare una definizione di censura (usando definizione da vocabolario)
    \item Formulare tesi su base censura - provenienza
    \item Creare un Dataset (almeno 30 prompt che garantiscono una validazione statistica abbastanza banale) 
    \textbf{Trovato Dataset}
    \item In che modo fare il Labeling? Sono fondamentali dei \textbf{criteri di valutazione della censura!!!}
    Creare una tabella con score che determina se risposta è censurata o meno (\textbf{definire che parametri} utilizzare per la valutazione
\end{itemize}

Almeno 4 persone per andare a valutare la coerenza. Le 4 cose fondamentali sono:
Prompt - Risposta - Labeling - Analisi della Risposta.

Per analizzare, usare HuggingChat (basarsi sui modelli di HuggingFace):
\begin{itemize}
    \item Cinesi:
    \begin{enumerate}
        \item QWEN
        \item DEEPSEEK
    \end{enumerate}
    \item Americani:
    \begin{enumerate}
        \item CHATGPT
        \item GEMINI
    \end{enumerate}
    \item Francesi (EUR):
    \begin{enumerate}
        \item Mistral AI
    \end{enumerate}
    \item Sconosciuti (sto cercando):
    \begin{enumerate}
        \item NousResearch
    \end{enumerate}
\end{itemize}

Se su HuggingChat ci sono modelli provenienti da altri Paesi, analizzarne il comportamento.

\subsection{Email 1}
In seguito al problema esposto di dover far mettere troppe crocette agli utenti, il professor Zingaro suggerisce di fare meno domande a più persone ma con overlap minimo.

\subsubsection{Come lo risolverebbe Daniele}
Dato l'intero dataset, ordinarlo in qualche modo\footnote{In ordine alfabetico delle domande magari} e successivamente, dividere il dataset in sotto insiemi non troppo grandi, l'id del chunk lo indichiamo con $c_{ID}$.
\newline
Dopodichè ogni persona che inizia il quiz, avrà anche questa un ID numerico progressivo $req_{ID}$.
\newline
L'utente risponderà alle domande t.c. $c_{ID} = MOD(req_{ID}, |CHUNKS|)$, dove $CHUNKS$ indica l'insieme dei sottoinsiemi e $|\bullet|$ indica la cardinalità di un insieme.
\newline
\textbf{Possibili problemi:} Potrebbero servire troppe persone. \newline Non è ideale far far rispondere le domande a gente a caso su internet, a causa, della potenziale presenza di troll.

\subsection{Nuova idea per Dataset}


\subsection{Link utili}
Alcune domande: \url{https://www.theguardian.com/technology/2025/jan/28/we-tried-out-deepseek-it-works-well-until-we-asked-it-about-tiananmen-square-and-taiwan} % Letto
Argomenti e tipi di censura: \url{https://americanedgeproject.org/new-report-chinese-ai-censors-truth-spreads-propaganda-in-aggressive-push-for-global-dominance/} % Letto
o 
\url{https://americanedgeproject.org/wp-content/uploads/2024/12/AEP-US-China-AI-Paper-2024-1.pdf} % Letto

dataset di prompt censurati dal Partito comunista cinese (CPP): \url{https://huggingface.co/datasets/promptfoo/CCP-sensitive-prompts} (vedere cosa fa promptfoo)

dataset di prompt a cui ChatGPT non risponde (censurati): \url{https://github.com/maxwellreuter/chatgpt-refusals} preso dal paper \url{https://arxiv.org/pdf/2306.03423}

Nota: Per i prompt di chatgpt riguardo gli eventi di guerra si possono aggiungere domande sulla questione israele-palestrina \url{https://www.instagram.com/p/CydbE5sutDQ/}

Conversazione riguardo la censura in generale e quella di ChatGPT: \url{https://news.ycombinator.com/item?id=42858797}
